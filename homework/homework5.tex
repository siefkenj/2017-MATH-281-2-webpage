\documentclass[letter]{article}
\usepackage{amsmath}
\usepackage{amsfonts}
\usepackage{amssymb}
\usepackage{ifthen}
\usepackage{fancyhdr}
\usepackage{enumitem}
\usepackage{mathrsfs}

%%%
% Set up the margins to use a fairly large area of the page
%%%
\oddsidemargin=.2in
\evensidemargin=.2in
\textwidth=6in
\topmargin=0in
\textheight=9.0in
\parskip=.07in
\parindent=0in
\pagestyle{fancy}

%%%
% Set up the header
%%%
\newcommand{\setheader}[6]{
	\lhead{{\sc #1}\\{\sc #2} ({\small \it \today})}
	\rhead{
		{\bf #3} 
		\ifthenelse{\equal{#4}{}}{}{(#4)}\\
		{\bf #5} 
		\ifthenelse{\equal{#6}{}}{}{(#6)}%
	}
}

%%%
% Set up some shortcut commands
%%%
\newcommand{\R}{\mathbb{R}}
\newcommand{\C}{\mathbb{C}}
\newcommand{\N}{\mathbb{N}}
\newcommand{\Z}{\mathbb{Z}}
\newcommand{\Proj}{\mathrm{proj}}
\newcommand{\Perp}{\mathrm{perp}}
\newcommand{\proj}{\mathrm{proj}}
\newcommand{\Span}{\mathrm{span}}
\newcommand{\Null}{\mathrm{null}}
\newcommand{\Rank}{\mathrm{rank}}
\newcommand{\mat}[1]{\begin{bmatrix}#1\end{bmatrix}}
\renewcommand{\d}{\mathrm{d}}

%%%
% This is where the body of the document goes
%%%
\begin{document}
	\setheader{Math 281-2}{Homework 5}{Not for turning in}{}{}{}
	\begin{enumerate}
		\item Consider the second order differential equation
			\[
				y''+y=0.
			\]
			\begin{enumerate}
				\item Find a power series solution for this differential equation.
				\item Use your knowledge of what solutions to this differential equation look
					like to rearrange your power series into some Taylor series 
					formulas for familiar functions.
			\end{enumerate}
		\item So far, we've only done power series solutions center at $0$.  That is,
			our guess for a power-series solution looks like $y=\sum_{i\geq 0} a_i x^i$.  This
			often times works, but sometimes it won't give all solutions because at $x=0$ some
			information is lost.
			
			Consider the differential equation
			\[
				xy''+y'+xy=0.
			\]
			\begin{enumerate}
				\item Classify this differential equation in terms of linear/autonomous/homogeneous/$n$th order etc.
				\item What dimension do you expect the space of all solutions to be?  That is, how many free
					parameters do you expect to have?
				\item Find a power series solution to this differential equation centered at $0$.  How many
					parameters do you have in your solution?
				\item The power series solution centered at $x_0$ is obtained by guessing a solutions of the
					form $y=\sum_{i\geq 0} a_i(x-x_0)^i$.  Find the power series solution to this 
					differential equation centered at $x_0=1$.  How many parameters do you have in your solution?
					Can you explain what happened in the previous part?
			\end{enumerate}
		\item We haven't spent a lot of time studying \emph{degenerate} differential equations/systems.  That is,
			differential equations/systems that have $0$ as an eigenvalue.  Let's fix that.
			\begin{enumerate}
				\item Find all solutions to the differential equation
					\begin{equation}
						\label{Eqdeg}
						y''+y'=0.
					\end{equation}
					Then, rewrite equation \eqref{Eqdeg} as a system in matrix form and find all flows.
				\item Consider the system of differential equations given in matrix form by
					\begin{equation}
						\label{Eqmatdeg}
						\mat{a'\\b'} = \mat{2&3\\6&9}\mat{a\\b}.
					\end{equation}
					Find the characteristic polynomial for equation \eqref{Eqmatdeg} and all eigenvalues.
				\item Can the differential equation \eqref{Eqmatdeg} be solved for all initial conditions
					$\mat{a'(0)\\b'(0)}=\mat{a_0\\b_0}$?  Explain.
				\item Write down all flows for equation \eqref{Eqmatdeg}.
			\end{enumerate}
		\item Use the Laplace transform to solve the following initial value problems.
			\begin{enumerate}
				\item $y''-10y'+9y=5t$ and $y(0)=-1$ and $y'(0)=2$.
				\item $2y''+3y'-2y=te^{-2t}$ and $y(0)=0$ and $y'(0)=-2$.
				\item $y''+3ty'-6y=2$ and $y(0)=0$ and $y'(0)=0$.
			\end{enumerate}
		\item Compute the following convolutions.
			\begin{enumerate}
				\item $t^2 * 1$
				\item $t*e^{-t}$
				\item $t^2*(t^2-1)$
				\item $\cos t*\cos t$
			\end{enumerate}
		\item Without computing, estimate where the asymptotes are in
			the analytic continuation of the following Laplace transforms.
			(In other words, if we did compute the Laplace transform
			and interpreted it as a formula ignoring the restrictions on 
			the domain, where would the asymptotes be).
			\begin{enumerate}
				\item $\mathscr{L}(e^t+e^{8t}-14e^{-14t})$
				\item $\mathscr L(1+(x-2)^2)$
				\item $\mathscr L(\sin^2(t)e^{-\pi t})$
				\item $\mathscr L(\sin t)$
				\item Compare your estimate for the asymptotes of $\mathscr L(\sin t)$ with
					the actual asymptotes of the formula.  Are they where you thought?
					(Don't forget about complex numbers.)

					Let $F(s) = \mathscr L(\sin t)$.  Explain why it makes sense that 
					\[
						\lim_{s\to 0} F(s) \neq \infty.
					\]

					Explain why 
					\[
						\lim_{s\to 0} F(i+s) = \pm\infty.
					\]
					You'll find Euler's formula helpful!
			\end{enumerate}
		\item We can do some really cool things with the Laplace transform that don't seem related to ODEs.
			The function 
			\[
			f(x) = \begin{cases}\frac{\sin x}{x}&\text{ if }x\neq 0\\1&\text{ if }x=0\end{cases}
			\]
			is called the \emph{sinc} function and shows up a lot in science and
			engineering.  We're going to find
			\[
				I = \int_0^\infty f(x)\d x.
			\]
			Since the sinc function resembles an alternating harmonic series, it's reasonable to
			assume that $I$ actually exists, and in fact using the alternating series
			test, we can prove that $I<\infty$ exists.  But integrating it directly is hard!
			\begin{enumerate}
				\item Write out the formula for $F(s)=\mathscr L(f(t))$, and compute 
					$\frac{\d f}{\d s}$.  Do you see see a Laplace transform
					that appears on your lookup table anywhere?
				\item You now have a formula for $\d F/\d s$, so if you find its antiderivative
					you'll obtain $F+k$ for some constant $k$.  Do so, and find
					the antiderivative that corresponds to $k=0$.  (Think about what this means.
					It's not as easy as not ``adding $k$'' when integrating.)  Hint: consider
					any values of $F(s)$ or any limits of $F(s)$ that you might actually know
					exactly.
				\item For what value of $s$ does $F(s)$ coincide with $I$?  Evaluate $F$ at
					that value to obtain $I$.
			\end{enumerate}

		\item There are several things that you need to know are associated
			with differential equations and solutions to ODEs.  Though I won't
			make you use these things, it is important that you recognize them as
			associated with differential equations so that you can look them up when
			you need them.

			Look up in your textbook (or Wikipedia or a source of your choice) the following
			items and read about them until they are cemented in your mind as associated
			with differential equations.
			\begin{enumerate}
				\item Separation of Variables
				\item Variation of Parameters
				\item Bessel Functions
			\end{enumerate}

	\end{enumerate}

\end{document}
