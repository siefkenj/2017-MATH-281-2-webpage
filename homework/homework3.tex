\documentclass[letter]{article}
\usepackage{amsmath}
\usepackage{amsfonts}
\usepackage{amssymb}
\usepackage{ifthen}
\usepackage{fancyhdr}
\usepackage{enumitem}

%%%
% Set up the margins to use a fairly large area of the page
%%%
\oddsidemargin=.2in
\evensidemargin=.2in
\textwidth=6in
\topmargin=0in
\textheight=9.0in
\parskip=.07in
\parindent=0in
\pagestyle{fancy}

%%%
% Set up the header
%%%
\newcommand{\setheader}[6]{
	\lhead{{\sc #1}\\{\sc #2} ({\small \it \today})}
	\rhead{
		{\bf #3} 
		\ifthenelse{\equal{#4}{}}{}{(#4)}\\
		{\bf #5} 
		\ifthenelse{\equal{#6}{}}{}{(#6)}%
	}
}

%%%
% Set up some shortcut commands
%%%
\newcommand{\R}{\mathbb{R}}
\newcommand{\C}{\mathbb{C}}
\newcommand{\N}{\mathbb{N}}
\newcommand{\Z}{\mathbb{Z}}
\newcommand{\Proj}{\mathrm{proj}}
\newcommand{\Perp}{\mathrm{perp}}
\newcommand{\proj}{\mathrm{proj}}
\newcommand{\Span}{\mathrm{span}}
\newcommand{\Null}{\mathrm{null}}
\newcommand{\Rank}{\mathrm{rank}}
\newcommand{\mat}[1]{\begin{bmatrix}#1\end{bmatrix}}
\renewcommand{\d}{\mathrm{d}}

%%%
% This is where the body of the document goes
%%%
\begin{document}
	\setheader{Math 281-2}{Homework 3}{Due Thursday, February 11}{}{}{}
	\begin{enumerate}
		\item Find all solutions to the following linear differential equations.
		\begin{enumerate}
			\item $y''-4y'+4y=0$
			\item $y''+4y=0$
			\item $y''-y=t^2$
			\item $y'''+4y''-5y'=0$
		\end{enumerate}

		\item Consider the differential equation
			\begin{equation}
				\label{EQ}
				y''-4y'+4y=0.
			\end{equation}
		\begin{enumerate}
			\item Find all values $k$ so that this equation solution of the form $e^{kt}$.
			\item Can you express all solutions to equation \eqref{EQ} in the form $Ae^{k_1t}+Be^{k_2t}$ for
				some fixed constants $k_1$ and $k_2$?  Why or why not?
			\item Find a $k$ such that $te^{kt}$ is a solution to equation \eqref{EQ}.
			\item Can you express all solutions to equation \eqref{EQ} in the form $Ae^{k_1t}+Bte^{k_2t}$ for
				some $k_1$ and $k_2$?  Explain.
			\item Can you express all solutions to equation \eqref{EQ} in the form $A(1+t)e^{k_1t}+B(1-t)e^{k_2t}$ for
				some $k_1$ and $k_2$?  Explain.
		\end{enumerate}

	\item The characteristic polynomial of a linear homogeneous differential equation $F(y)=0$ is
		the polynomial $p(k)$ so that $F(e^{kt})=p(k)e^{kt}$.  (Recall here that $F$ a linear
		function.  For example, maybe $F(y) = y''-2y$.)
		\begin{enumerate}
			\item For each linear differential equations from part 1, write the characteristic polynomial  (if it isn't
				homogeneous, consider the corresponding homogeneous equation).
			\item Consider a second-order linear homogeneous ODE with 
				characteristic polynomial $p(k)$ having roots $\alpha$ and $\beta$.  Write the equation for
				this ODE.
			\item Consider a second-order linear homogeneous ODE with 
				characteristic polynomial $p(k)$.
				Suppose that $p(k)$ is two roots $\alpha,\beta $ with $\alpha\neq \beta$.
				Show that $Ae^{\alpha t}+Be^{\beta t}$ is always a solution.  Further, show that
				all solutions to this differential equation can be written in this way.
				
			\item Consider a second-order linear homogeneous ODE with 
				characteristic polynomial $p(k)$ having one repeated root $\alpha$.  
				Show that $Ae^{\alpha t}+Bte^{\alpha t}$ is always a solution.  Further, show that
				all solutions to this differential equation can be written in this way.
			\item Consider a second-order linear homogeneous ODE with 
				characteristic polynomial $p(k)$ having two distinct roots $\alpha,\beta$.  
				Show that there is no $\gamma$ such that $te^{\gamma t}$ is a solution.
			\item Consider a third-order linear homogeneous ODE with 
				characteristic polynomial $p(k)=(k-2)^3$.  
				Write the equation for this ODE.  Then, find all solutions.
		\end{enumerate}
	\item Consider $y''+y=0$.
		\begin{enumerate}
			\item Show that for any initial value problem, you can find $A,B\in \C$ so that
				\[
					Ae^{i\theta}+Be^{-i\theta}
				\]
				is a solution.  Could you solve any IVP if you restrict $A,B\in\R$?
			\item Use part (a) along with Euler's formula and the fact that $\sin$ and $\cos$ are solutions
				to this differential equation to come up with a formula for $\sin$ and $\cos$ as a linear
				combination of complex exponentials.
			\item Show that $A\cos \theta+B\sin\theta$ also solves every initial value problem.
			\item Can you get any additional solutions by considering $A\cos \theta+B\sin\theta +Ae^{i\theta}+Be^{-i\theta}$?
				Explain.
			\item Use your formula from part (b) to find $\arccos$ and $\arcsin$ in terms of complex numbers and complex
				logarithms.  (Hint: $e^{2t}=(e^t)^2$ and $e^te^{-t}=1$).
		\end{enumerate}
	\item The matrices are coming!  Read about \emph{matrix multiplication} in the source of your choice (the Evans
		text talks about it) and how to compute $2\times 2$ and $3\times 3$ determinants.
		\begin{enumerate}
			\item Compute $\mat{1&2\\3&4\\5&6}\mat{1&1&1\\1&-1&1}$, $\det\left(\mat{1&1\\2&2}\right)$,
				and $\mat{1&1&1\\1&-1&-1\\-1&-1&1}^2$.
			\item Let $\vec a=(1,2,3)$ and $\vec b=(1,2,2)$.  Find matrices $A$ and $B$
				so that $\vec a\cdot \vec b$ is equivalent to the matrix product $AB$.
				What are the dimensions of $A$ and $B$?
			\item Let $M=\mat{1&2\\1&1}$, and consider the function $F_M:\R^2\to \R^2$ defined by 
				\[
					F_M(a,b) = \mat{1&2\\1&1}\mat{a\\ b}.
				\]
				Show that $F_M$ is linear.
		\end{enumerate}

	\end{enumerate}

\end{document}
