\documentclass[letter]{article}
\usepackage{amsmath}
\usepackage{amsfonts}
\usepackage{amssymb}
\usepackage{ifthen}
\usepackage{fancyhdr}
\usepackage{enumitem}

%%%
% Set up the margins to use a fairly large area of the page
%%%
\oddsidemargin=.2in
\evensidemargin=.2in
\textwidth=6in
\topmargin=0in
\textheight=9.0in
\parskip=.07in
\parindent=0in
\pagestyle{fancy}

%%%
% Set up the header
%%%
\newcommand{\setheader}[6]{
	\lhead{{\sc #1}\\{\sc #2} ({\small \it \today})}
	\rhead{
		{\bf #3} 
		\ifthenelse{\equal{#4}{}}{}{(#4)}\\
		{\bf #5} 
		\ifthenelse{\equal{#6}{}}{}{(#6)}%
	}
}

%%%
% Set up some shortcut commands
%%%
\newcommand{\R}{\mathbb{R}}
\newcommand{\N}{\mathbb{N}}
\newcommand{\Z}{\mathbb{Z}}
\newcommand{\Proj}{\mathrm{proj}}
\newcommand{\Perp}{\mathrm{perp}}
\newcommand{\proj}{\mathrm{proj}}
\newcommand{\Span}{\mathrm{span}}
\newcommand{\Null}{\mathrm{null}}
\newcommand{\Rank}{\mathrm{rank}}
\newcommand{\mat}[1]{\begin{bmatrix}#1\end{bmatrix}}
\renewcommand{\d}{\mathrm{d}}

%%%
% This is where the body of the document goes
%%%
\begin{document}
	\setheader{Math 281-2}{Homework 2}{Due Thursday, January 19}{}{}{}
	\begin{enumerate}
		\item A first order differential equation is called \emph{separable} if
			it can be written in the form $f(y)\frac{\d y}{\d x}= h(x)$.  Separable
			equations are easy to implicitly solve. By integrating both sides, we have
			$
				\int f(y)\frac{\d y}{\d x}\d x= \int h(x)\d x.
			$ Applying the chain rule to the left hand side gives us
			\[
				\int f(y)\d y = \int h(x)\d x.
			\]

			For each differential equation, state whether or not it is separable.
			If it is separable, find the set of implicit solutions.
			Further, find explicit solutions (i.e., solutions
			where $y$ is a function of $x$) to the initial value problem $y(x_0)=y_0$
			for every applicable $(x_0,y_0)$.  (For example, if the implicit
			solution were $x^2+y^2=K$, then the solution would be $y=\sqrt{(x_0^2+y_0^2)-x^2}$
			if $y_0> 0$ and $y=-\sqrt{(x_0^2+y_0^2)-x^2}$ if $y_0<0$.  If you encounter
			a similar situation, you must include both such solutions and describe which initial
			conditions give rise to which solution.)
			\begin{enumerate}
				\item $y' = \frac{1-2y}{x}$
				\item $y'=\frac{-xy}{x+1}$
				\item $y' = 3\sqrt[3]{y^2}$
				\item $xy'+y=y^2$
			\end{enumerate}


		\item Consider the differential equation $y'=Ky-y^3$ where $K$ is constant.
		\begin{enumerate}
			\item Draw a slope field for the differential equation for your choice of $K$.  Make sure it's
				detailed enough for you to see what's going on.
			\item Partition the set of initial conditions $(x_0,y_0)$ into sets that
				give rise to ``similar looking'' solutions and describe the types
				of solutions you get from each set.  Warning: your partitions may depend on $K$!
			\item Let $\vec x_0=(x_0,y_0)$ and $\vec x_0^*=(x_0^*,y_0^*)$ with $\|\vec x_0-\vec x_0^*\|<\varepsilon$.
				Let $y(x)$ and $y^*(x)$ be solutions to the respective initial value problems.
				The initial condition $\vec x_0$ is called \emph{bi-stable} if $|y(x)-y^*(x)|<C\varepsilon$
				for some fixed $C$.  Otherwise it is called \emph{unstable}.  
				It is called \emph{forward stable} if $|y(x)-y^*(x)|<C\varepsilon$ when $x\geq x_0$
				and \emph{backwards stable} if $|y(x)-y^*(x)|<C\varepsilon$ when $x\leq x_0$ (so stable implies
				both forward and backwards stable).

				Identify each region of of initial conditions as forward/backward/bi stable or unstable.
			\item  For $K=2$, use Euler's method with 5 steps
				to estimate $y(5)$ where $y$ is a solution
				to the initial value problem $(0,1)$. Plot your estimate.  
				What is going on?
			\item Repeat your calculation for $K=2$ using Euler's method with 50 steps.
				(You can use a computer for the 50 steps.)

			\item Use Euler's method and this differential equation (with an appropriate value for $K$) to approximate
				$\sqrt{7}$.
		\end{enumerate}

	\end{enumerate}

\end{document}
